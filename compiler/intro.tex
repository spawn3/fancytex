\chapter{缘起}

2009年底,一个偶然的机会,知道了Erlang编程语言。
随便Google了一下,据说在函数式编程,并行,分布式和容错方面有不俗的表现。
趁着春节假期的机会,写了一个玩具项目EFS(Erlang File System),算是对语言的语法有个感性的认识了。

2010年初,三月份的样子,公司要做一个虚拟机管理系统,联系到Erlang在分布式和容错
方面的表现,就建议采用Erlang作为实现语言,经过一些讨论,得以确定下来,这是用它做得第一个正式项目。

经过这半年来的使用,有语法上的别扭,有函数式编程的不适应,也有通过很少代码完成一个特性时那种畅快感。
最近琢磨着对它进行一番深入的剖析,一来可以提高自己掌控大系统的能力,对各个不同的技术方向也是很好的学习机会,
二来提高一下写作的技能。思念及此,不觉心动,立定决心,把这个工作做下去。

Erlang的启动过程

Erlang的进程模型和消息传递

分布式Erlang

Erlang内置对象

Erlang虚拟机


