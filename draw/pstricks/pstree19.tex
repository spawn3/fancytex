\documentclass[12pt,a4paper]{article}
\usepackage[T1]{fontenc}
\usepackage[utf8]{inputenc}
\usepackage[ngerman]{babel}
\usepackage{mathpazo}
\usepackage{pst-tree}
\renewcommand\TR[2][2.1cm]{\Tr{\tabular{p{#1}}#2\endtabular}}
\parindent0em

\begin{document}

\footnotesize
\psset{levelsep=3cm,treesep=4mm,nodesep=3pt,linecolor=blue}
\pstree[treemode=R,thislevelsep=2cm]{\Tr{Mobilitätsmechnismus}}{%
	\pstree{\TR[1.25cm]{Schwache\\Mobilität}}{%
		\pstree[thislevelsep=3.5cm]{\TR{Senderinitiierte\\Mobilität}}{%
			\TR[2.7cm]{Ausführen im\\Zielprozess}
			\TR[2.7cm]{Ausführen in\\eigenem Prozess}
		}
		\pstree[thislevelsep=3.5cm]{\TR{Empfänger-\\initiierte\\Mobilität}}{%
			\TR[2.7cm]{Ausführen im\\Zielprozess}
			\TR[2.7cm]{Ausführen in\\eigenem Prozess}
		}
	}
	\pstree{\TR[1.25cm]{Starke\\Mobilität}}{%
		\pstree[thislevelsep=3.5cm]{\TR{Senderinitiierte\\Mobilität}}{%
			\TR[2.7cm]{Prozess migrieren}
			\TR[2.7cm]{Prozess klonen}
		}
		\pstree[thislevelsep=3.5cm]{\TR{Empfänger-\\initiierte\\Mobilität}}{%
			\TR[2.7cm]{Prozess migrieren}
			\TR[2.7cm]{Prozess klonen}
		}
	}
}
\end{document}
