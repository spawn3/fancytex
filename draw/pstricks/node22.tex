\documentclass[11pt]{article}% Michael Sharpe
\usepackage{pstricks}
\usepackage{pstricks-add}
\pagestyle{empty}
%macro to make polyline joining sequence of nodes
\def\psnline#1(#2,#3)#4{\xdef\tmp{#1}\multido{\i=#2+1}{#3}%
{\xdef\tmp{\tmp(#4\i)}}\expandafter\psline\tmp}
%\psnline[linecolor=red,arrows=->](0,101){P} expands to
% \psline[linecolor=red,arrows=->](P0)(P1)...(P100)
% assuming P0,...,P100 are previously defined nodes.

\makeatletter
% build the linear combination #3[1]]*#1+#3[[2]]*#2=#4
\def\psLCNodeVar(#1)(#2)(#3)#4{%
  \pst@getcoor{#1}\pst@tempA%
  \pst@getcoor{#2}\pst@tempB%
  \pnode(#3){tmpn@de}
  \pnode(!%
    \pst@tempA /YA exch \pst@number\psyunit div def
    /XA exch \pst@number\psxunit div def
    \pst@tempB /YB exch \pst@number\psyunit div def
    /XB exch \pst@number\psxunit div def %stack now empty
    \psGetNodeCenter{tmpn@de}\space
    XA tmpn@de.x mul XB tmpn@de.y mul add
    YA tmpn@de.x mul YB tmpn@de.y mul add){#4}%
}
\makeatother
\begin{document}
\SpecialCoor
% Given two sequences of pnodes, P0...Pn, Q0...Qn, draw polyline joining
% some interpolating points R0...Rn
\begin{pspicture}[showgrid=true](-5,-3)(5,3)
\multido{\i=0+1}{361}{%
\pnode(! \i\space dup cos 3 mul exch sin 2 mul){P\i}%points on inner  ellipse
\pnode(! \i\space dup cos 5 mul exch sin 3 mul){Q\i}%points on outer  ellipse
\pnode(! \i\space 3 mul cos dup mul dup 1 sub neg){C\i}%coefficients
\psLCNodeVar(P\i)(Q\i)(C\i){R\i}% define interpolating points
}
\psnline(0,361){P}\psnline(0,361){Q}
\psnline[linewidth=1.7pt,linecolor=red](0,361){R}
\end{pspicture}
\end{document}




\listfiles
\documentclass[11pt]{article}
%\usepackage{amsmath, amssymb}
\usepackage[dvipsnames, svgnames]{pstricks}
\usepackage{pst-solides3d}

\begin{document}

\begin{pspicture}(-6,-3)(6,3)
\psset{unit=1.5}
\defFunction{mobius}(u,v)
  {2 u v Cos mul add 2 v mul Cos mul}
  {2 u v Cos mul add 2 v mul Sin mul}
  {u v Sin mul}
\defFunction{centercircle}(t){2 t mul Cos 2 mul}{2 t mul Sin 2 mul}{0}
\psset[pst-solides3d]{viewpoint=20 10 5,Decran=40,lightsrc=10 20 10}
\psSolid[object=surfaceparametree, linewidth=0.5\pslinewidth,
  base=-0.5 0.5 0 pi, fillcolor=lightgray!30, incolor=white,
  function=mobius,ngrid=.05]
\psSolid[object=courbe, range= 0 pi,
linewidth=1.75pt, linecolor=blue,
function=centercircle, ngrid=1,r=0] 

\end{pspicture}

\end{document}





\documentclass{scrartcl}
\usepackage[ngerman]{babel}
\usepackage{array}
\usepackage{tabularx}
\renewcommand\tabularxcolumn[1]{b{#1}}

\begin{document}

 \begin{minipage}{8cm}
 \begin{tabularx}{\linewidth}{r@{\quad}X@{}l}
 01 & Damit die Zeile nicht so lang ist, 
      erstellen wir eine Minipage\dotfill &  03:45 \\
 02 & Eine kurze Zeile\dotfill &  05:34 \\
 03 & Damit die Zeile nicht so lang ist, 
      erstellen wir eine Minipage\dotfill & 03:45 
\end{tabularx}
\end{minipage}

\end{document}



\documentclass{article}
\usepackage[T1]{fontenc}
\usepackage{pstricks-add}
\parindent=0pt

\begin{document}

\psset{xunit=8,yunit=3}
\begin{pspicture}(-0.5,-1.5)(1,1.5)
  \psgrid[griddots=10,gridlabels=0pt](0,-1)(1,1)
  \psaxes[Dx=0.2,Oy=-1,Dy=0.4,tickstyle=top,
          axesstyle=frame](0,-1)(1,1)
 \rput(0.5,1.3){Curves example}
 \rput(-0.3,0){\shortstack{%
   \textcolor{blue}{$\dfrac{x}{(x + 0.3)}$}\\
      \textcolor{red}{$\sin (10 \times x) / 2$}}}
  \rput(0.5,-1.3){\shortstack{$x$\\Sample plots}}
  \psplot[linecolor=red,linewidth=2pt]% 1 radian = 57.296 degrees
         {0}{1}{x 10 mul 57.296 mul sin 0.5 mul}  % sin(10 x) / 2
  \psplot[linecolor=blue,linewidth=2pt]%
      {0}{1}{x 0.3 x add div} % x / (x + 0.3)
\end{pspicture}

\end{document}

\readdata{\data}{2.txt}
%\readdata{\dataA}{22.txt}

\pstScalePoints(1,1){9.3333 sub neg 7 add }{}
\psset{llx=-5mm,lly=-5mm}
\begin{psgraph}[yAxis=false,Dx=0.2,dx=-0.2,Ox=9.4](7,0)(9.4,4){\linewidth}{7cm}
\listplot[linecolor=red,linewidth=0.6pt]{\data}
%\listplot[linecolor=blue,linewidth=0.6pt]{\dataA}
\end{psgraph}


\psset{xunit=6.3cm,yunit=1cm}
\begin{pspicture}(7,0)(9.4,4)
\psaxes[yAxis=false,Dx=0.2,dx=-0.2,Ox=9.4](7,0)(9.4,4)
\listplot[linewidth=0.5pt]{\data}
%\listplot[linewidth=0.5pt]{\dataII}
\uput[-90](1,-0.5){$\delta$ / ppm}
\end{pspicture}

\end{document}



     coorType 4 le { % coorType |/_ with a 1/sqrt(2) shortend x-axis 
and 135 degrees
       /x2D x -1 mul 2 y mul add z 0 mul add  def %% z 0 mul --> not 
really useful
       /y2D x -1 mul -0.5 y mul add z 2 mul add def
       exit } if