\documentclass{article}% voss 2006-08-28
\usepackage{multido,pstricks,pst-node}
\def\Square{\rule{1cm}{1cm}}
\SpecialCoor
\begin{document}

\vspace*{3cm}
\pspicture[showgrid=true](5,1)%
\Rnode{A}{\Square}\hspace{3cm}\Rnode{B}{\Square}
\ncline[linecolor=red,linewidth=2pt,arrowscale=2,arrows=->]{A}{B}
\endpspicture

\bigskip
\pspicture[showgrid=true](5,4)%
\rput[lb](0,0){\Rnode{A}{\Square}\hspace{3cm}\Rnode[vref=3cm]{B}{\Square}}
\psframe*[linecolor=black!20](4,3)(5,4)
\pcline[linecolor=red,linestyle=dotted,linewidth=2pt](A)(B)
\ncline[linecolor=red,linewidth=2pt,arrowscale=2,arrows=->]{A}{B}
\pnode(4,0.7ex){C}
\ncline[linecolor=blue,linewidth=2pt,arrowscale=2,arrows=->]{A}{C}
\pcline{->}(4.5,0.5)(4.5,3.5)\uput[0](4.5,2){vref}
\pcline[linecolor=white](4.5,0.5)(4.5,1)
\endpspicture


\bigskip
\pspicture[showgrid=true](7,4)%
\rput[lb](0,0){\Rnode{A}{\Square}\hspace{3cm}\Rnode[href=4,vref=3cm]{B}{\Square}}
\psframe*[linecolor=black!20](4,3)(5,4)
\psframe*[linecolor=black!20](6,3)(7,4)
\psline[arrows=->](4.5,3.5)(6.5,3.5)\uput[-90](5.5,3.5){href}
\pcline[linecolor=red,linestyle=dotted,linewidth=2pt](A)(B)
\ncline[linecolor=red,linewidth=2pt,arrowscale=2,arrows=->]{A}{B}
\pnode(4,0.7ex){C}
\ncline[linecolor=blue,linewidth=2pt,arrowscale=2,arrows=->]{A}{C}
\pcline{->}(4.5,0.5)(4.5,3.5)\uput[0](4.5,2){vref}
\pcline[linecolor=white](4.5,0.5)(4.5,1)
\endpspicture

\clearpage
\begin{verbatim}
...
\usepackage{multido,pstricks,pst-node}
\def\Square{\rule{1cm}{1cm}}
\SpecialCoor

\begin{document}

\vspace*{3cm}
\pspicture[showgrid=true](5,1)%
\Rnode{A}{\Square}\hspace{3cm}\Rnode{B}{\Square}
\ncline[linecolor=red,linewidth=2pt,arrowscale=2,arrows=->]{A}{B}
\endpspicture

\bigskip
\pspicture[showgrid=true](5,4)%
\rput[lb](0,0){\Rnode{A}{\Square}\hspace{3cm}\Rnode[vref=3cm]{B}{\Square}}
\psframe*[linecolor=black!20](4,3)(5,4)
\pcline[linecolor=red,linestyle=dotted,linewidth=2pt](A)(B)
\ncline[linecolor=red,linewidth=2pt,arrowscale=2,arrows=->]{A}{B}
\pnode(4,0.7ex){C}
\ncline[linecolor=blue,linewidth=2pt,arrowscale=2,arrows=->]{A}{C}
\pcline{->}(4.5,0.5)(4.5,3.5)\uput[0](4.5,2){vref}
\pcline[linecolor=white](4.5,0.5)(4.5,1)
\endpspicture


\bigskip
\pspicture[showgrid=true](7,4)%
\rput[lb](0,0){\Rnode{A}{\Square}\hspace{3cm}\Rnode[href=4,vref=3cm]{B}{\Square}}
\psframe*[linecolor=black!20](4,3)(5,4)
\psframe*[linecolor=black!20](6,3)(7,4)
\psline[arrows=->](4.5,3.5)(6.5,3.5)\uput[-90](5.5,3.5){href}
\pcline[linecolor=red,linestyle=dotted,linewidth=2pt](A)(B)
\ncline[linecolor=red,linewidth=2pt,arrowscale=2,arrows=->]{A}{B}
\pnode(4,0.7ex){C}
\ncline[linecolor=blue,linewidth=2pt,arrowscale=2,arrows=->]{A}{C}
\pcline{->}(4.5,0.5)(4.5,3.5)\uput[0](4.5,2){vref}
\pcline[linecolor=white](4.5,0.5)(4.5,1)
\endpspicture
\end{verbatim}

\end{document}

\multido{\iA=-15+1}{30}{\rput(4,0){\pspicture(5,7)%
  \Rnode[href=\iA,vref=3cm]{A}{\psframebox[framesep=0pt]{\hspace*{1cm}}}\hspace{2cm}\Rnode{B}{o}
  \ncline{A}{B}\endpspicture}}%


\endinput

\documentclass[12pt,ngerman]{article}
\usepackage[T1]{fontenc}
\usepackage[latin9]{inputenc}
\usepackage{pstricks,pst-node,pst-plot}

\begin{document}

\begin{pspicture}(-0.5,-0.5)(3,2)
  \psaxes[axesstyle=none,labels=none,ticks=none]{->}(3,2)
\end{pspicture}

\def\test{13.444}
\leavevmode\typeout{\numexpr\test}
\rput(\numexpr\test,0){huhu}



\rnode{A}{\psshadowbox{\parbox{4cm}{%
\[
\left( \frac{p_{11}p_{ij}}{p_{1j}p_{i1}} \right)\left(\frac{p_{11}p_{i'j'}}{p_{1j'}p_{i'1}}\right)
\]}}}% 
\hspace*{4cm}%
\rnode{B}{\psshadowbox{\parbox{2cm}{%
	Hier kann irgendeine Erkl�rung oder sonstwas stehen}}}
\ncline[arrows=->,arrowscale=2]{A}{B}
\naput{Und hier Text}

\end{document}

\documentclass[a4paper]{article}

\usepackage[T1]{fontenc}
\usepackage[latin1]{inputenc}
%\usepackage[esperanto,frenchb]{babel}
%\usepackage[pdftex]{graphicx}
%\usepackage{lmodern}
\usepackage{ifthen}
\usepackage{pstricks,pst-all}
\usepackage{pgffor}

\title{}
\author{}
\date{}

\definecolor{brun}{cmyk}{.4 .7 .7 0}

\newcommand{\bille}[2]{\pspolygon[linearc=.2, linecolor=darkgray,
  fillstyle=solid, fillcolor=#2,origin={#1}](-6,0)(-1.5,3.7)(1.5,3.7)(6,0)(1.5,-3.7)(-1.5,-3.7)}

\newcommand{\support}[1]{%
  \psline[linewidth=2.65, linecolor=brun](#1,-0.1)(#1,51.1)
  \psline[linewidth=2.7, linecolor=white,origin={#1,0}])(-6.95,37.7)(6.95,37.7)
  \psline[linewidth=0.6, linecolor=black,origin={#1,0}](-6.95,36.4)(6.95,36.4)
  \psline[linewidth=0.6, linecolor=black,origin={#1,0}](-6.95,39)(6.95,39)%
}

\newcommand{\tige}[3]{%
  \support{#1}
  \ifthenelse{#2>4}{\bille{#1,43.1}{yellow}}{\bille{#1,47.25}{yellow}}
  \ifthenelse{\(#2=0\)\or\(#2=5\)}{\foreach \m in {3.7,11.1,18.5,25.9}
    {\bille{#1,\m}{yellow}}}{%
      \ifthenelse{\(#2=1\)\or\(#2=6\)}{\foreach \m in {3.7,11.1,18.5,32.3}
    {\bille{#1,\m}{yellow}}}{%
      \ifthenelse{\(#2=2\)\or\(#2=7\)}{\foreach \m in {3.7,11.1,24.9,32.3}
{\bille{#1,\m}{yellow}}}{%
\ifthenelse{\(#2=3\)\or\(#2=8\)}{\foreach \m in {3.7,17.55,24.9,32.3}
{\bille{#1,\m}{yellow}}}{%
\ifthenelse{\(#2=4\)\or\(#2=9\)}{\foreach \m in {10.15,17.55,24.9,32.3}
{\bille{#1,\m}{yellow}}}{%
}}}}}
\ifthenelse{#3=1}{\pscircle*(#1,37.7){0.8}}{}
}

\newcommand{\cadre}[1]{\psframe[linewidth=2,framearc=.1](-2.1,-2.1)(#1,53.05)}

\begin{document}

\psset{unit=1mm}

\pspicture(43.6,53.05)                         % 43.6   = 2.05 +(3*13.85) <= 3 tiges
\tige{6.925}{9}{0}                             %  6.925 = (1*13.85) -6.925 <= tige 1
\tige{20.775}{0}{0}                            % 20.775 = (2*13.85) -6.925 <= tige 2
\tige{34.625}{6}{1}                            % 34.625 = (2*13.85) -6.925 <= tige 3
\cadre{43.6}                                   % 43.6   = 2.05 +(3*13.85) <= 3 tiges
\endpspicture

\end{document}



%\documentclass[12pt,a4paper]{article}
%\usepackage{lmodern}
\documentclass[minion]{ttct}
\usepackage[T1]{fontenc}
\usepackage{pstricks,longtable,multido,textcomp}
\begin{document}


\pspolygon[arrowscale=5,arrows=<->](1,1)(2,3)(4,0)

\vspace{2cm}

\def\Loption#1{\texttt{#1}}
\def\Lcs#1{\texttt{\textbackslash#1}}

{\psset{dotscale=1.5}
\def\multiDot#1{\multido{\nA=0+1}{5}{\psdot[dotstyle=#1]\kern1em}}
\def\cmultiDot#1{\multido{\nA=0+1}{5}{\psdot*[dotstyle=#1]\kern1em}}
\begin{longtable}{@{}l@{\kern2em}cc@{}}
\caption{Summary of all dot styles}\label{tab:dotstyle}\\[-5pt]
\emph{name} & \Lcs{psdot}    & \Lcs{psdot*}\\\hline
\endfirsthead
\emph{name} & \Lcs{psdot}    & \Lcs{psdot*}\\\hline
\endhead
\Loption{*}             & \multiDot{*}        & \cmultiDot{*} \\        
\Loption{o}             & \multiDot{o}         & \cmultiDot{o}\\
\Loption{Bo}            & \multiDot{Bo}        & \cmultiDot{Bo}\\
\Loption{x}             & \multiDot{x}        & \cmultiDot{x}\\
\Loption{+}             & \multiDot{+}        &\cmultiDot{+} \\
\Loption{B+}            & \multiDot{B+}        & \cmultiDot{B+}\\
\Loption{asterisk}      & \multiDot{asterisk}        & \cmultiDot{asterisk}\\
\Loption{Basterisk}     & \multiDot{Basterisk}        & \cmultiDot{Basterisk}\\
\Loption{oplus}         & \multiDot{oplus}     & \cmultiDot{oplus}\\
\Loption{otimes}        & \multiDot{otimes}    & \cmultiDot{otimes}\\
\Loption{|}             & \multiDot{|}    & \cmultiDot{|}\\
\Loption{square}        & \multiDot{square}    & \cmultiDot{square}\\
\Loption{Bsquare}       & \multiDot{Bsquare}   & \cmultiDot{Bsquare}\\
\Loption{square*}       & \multiDot{square*}    & \cmultiDot{square*}\\
\Loption{diamond}       & \multiDot{diamond}   & \cmultiDot{diamond}\\
\Loption{Bdiamond}      & \multiDot{Bdiamond}  & \cmultiDot{Bdiamond}\\
\Loption{diamond*}      & \multiDot{diamond*}    & \cmultiDot{diamond}\\
\Loption{triangle}      & \multiDot{triangle}  & \cmultiDot{triangle}\\
\Loption{Btriangle}     & \multiDot{Btriangle} & \cmultiDot{Btriangle}\\
\Loption{triangle*}     & \multiDot{triangle*} & \cmultiDot{triangle*}\\
\Loption{pentagon}      & \multiDot{pentagon}  & \cmultiDot{pentagon}\\
\Loption{Bpentagon}     & \multiDot{Bpentagon} & \cmultiDot{Bpentagon}\\
\Loption{pentagon*}     & \multiDot{pentagon*}& \cmultiDot{pentagon*}\\
\end{longtable}
}

\end{document}
\def\pshlabel#1{\footnotesize#1}
\let\psvlabel\pshlabel

\begin{pspicture}(4,1)
  \psaxes[Dy=0.25]{->}(4,1)
\end{pspicture}



\end{document}