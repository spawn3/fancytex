\documentclass[12pt,a4paper]{article}% Walter Schmidt
\usepackage[T1]{fontenc}\usepackage{textcomp}
\usepackage{mathpazo}
\usepackage{courier}
\usepackage{geometry,url,ngerman}

\usepackage[inactive]{pst-pdf}

\parindent0pt
\pagestyle{empty}

\usepackage{pstricks,pst-node}
\newcounter{leaves}
\newcounter{directories}

\newenvironment{directory}[2][\linewidth]%
% Startet neues Verzeichnis 
% und produziert eine Minipage der angeg. Breite.
% Syntax: \begin{directory}[width]{text}
% text muss in eine \parbox der angegebenen Breite passen;
% wenn keine Breite angegeben ist, wird \linewidth angenommen.
{%
\setcounter{leaves}{0}%
\addtocounter{directories}{1}
\edef\directoryname{D\thedirectories}
\begin{minipage}[t]{#1}% <-------- !!!
  \setlength{\parindent}{\linewidth}
  \addtolength{\parindent}{-\dirshrink\parindent}
  \parskip0pt%
  \noindent
  \Rnode[href=-\dirshrink]{\directoryname}{\parbox[t]{#1}{#2}}%
  \par
}  
{\end{minipage}}

% !!! --> Problem:  
% Wegen [t] stimmt der Zeilenabstand _nach_ der minipage nicht.
% Der Referenzpunkt eines Knoten muss aber in der _ersten_ Zeile 
% liegen, mehrzeilige Knoten, also Unterverzeichnisse, mit ihrer
% ersten Zeile im Dateibaum verankert weren.

\newcommand{\file}[2][]{%
% Fuer einen einzelnen Eintrag innerhalb der directory-Umgebung.
% Das Argument darf seinerseits eine directory-Umgebung sein.
  \addtocounter{leaves}{1}%
  \edef\leaflabel{L\theleaves\directoryname}%
  \par
  \Rnode{\leaflabel}{\parbox[t]{\dirshrink\linewidth}{#2\hfill#1}}%
  \ncangle[angleA=270,angleB=180,armB=0,nodesep=1pt]
    {\directoryname}{\leaflabel}%
  % \typeout{\directoryname,\leaflabel}% Debugging
\par}

\newcommand{\dirshrink}{.95} 
% relative Verringerung der Breite der Verzeichniseintraege 
% pro Stufe


\begin{document}

The draft directory of \url{fontinst}:
\begin{verbatim}
doc/
  manual/
  fontinst.aux
  fontinst.log
  fontinst.pdf
  fontinst.tex
  fontinst.toc
  intro98.tex
  ltxguide.cfg
  roadmap.eps
encspecs.zip
examples.zip
inputs.zip
latex.zip
README
source.zip
test.zip
\end{verbatim}


\medskip

\dots\ and what can be with \verb+pst-tree+

\def\url#1{#1}

\begin{postscript}
\begin{directory}{\url{fontinst}}
\file{\begin{directory}{\url{doc/}}
  \file{\begin{directory}{\url{manual/}}
    \file[ auxiliary file]{\url{fontinst.aux}}
    \file{\url{fontinst.log}}
    \file{\url{fontinst.pdf}}
    \file{\url{fontinst.tex}}
    \file[ table of contents ]{\url{fontinst.toc}}
    \file{\url{intro98.tex }}
    \file{\url{ltxguide.cfg}}
    \file{\url{roadmap.eps }}
  \end{directory}}
\file{\url{encspecs.zip }}
\end{directory}}
\file{\url{examples.zip }}
\file{\url{inputs.zip   }}
\file{\url{latex.zip    }}
\file{\url{README       }}
\file{\url{source.zip   }}
\file{\url{test.zip     }}
\end{directory}
\end{postscript}

\end{document}

