\documentclass[12pt,a4paper]{article}
\usepackage[latin1]{inputenc}
\usepackage[T1]{fontenc}
\usepackage{mathpazo}
\usepackage{pstricks,pst-node}

\begin{document}
\psset{nodesep=3pt}

\newrgbcolor{lila}{0.6 0.2 0.5}
\newrgbcolor{darkyellow}{1 0.9 0}
Die Bindungsenergie im Tr�pfchenmodell setzt sich aus
folgenden Teilen zusammen:
\begin{itemize}
\item dem \rnode{b}{Oberfl�chenanteil}
\item Dem \rnode{a}{Volumenanteil},\\[1cm]
\def\xstrut{\vphantom{\frac{(A)^1}{(B)^1}}}
\begin{equation}
E =
\rnode[t]{ae}{\psframebox*[fillcolor=darkyellow,
  linestyle=none]{\xstrut a_vA}} +
\rnode[t]{be}{\psframebox*[fillcolor=lightgray,
  linestyle=none]{\xstrut -a_fA^{2/3}}} +
\rnode[t]{ce}{\psframebox*[fillcolor=green,
  linestyle=none]{\xstrut -a_c\frac{Z(Z-1)}{A^{1/3}}}} +
\rnode[t]{de}{\psframebox*[fillcolor=cyan,
  linestyle=none]{\xstrut -a_s\frac{(A-2Z)^2}{A}}} +
\rnode[t]{ee}{\psframebox*[fillcolor=yellow,
  linestyle=none]{\xstrut E_p}}
\end{equation}\\[0.25cm]
\item dem \rnode{c}{Coulomb-Anteil}
\item der \rnode{d}{Symmetrieenergie}
\item sowie einem \rnode{e}{Paarbildungsbeitrag}.
\end{itemize}

\nccurve[angleA=-90,angleB=90]{->}{a}{ae}
\nccurve[angleB=45]{->}{b}{be}
\nccurve[angleB=-90]{->}{c}{ce}
\nccurve[angleB=-90]{->}{d}{de}
\nccurve[angleB=-90]{->}{e}{ee}


\begin{verbatim}
\usepackage{mathpazo}
\usepackage{pstricks,pst-node}

\begin{document}
\psset{nodesep=3pt}

\newrgbcolor{lila}{0.6 0.2 0.5}
\newrgbcolor{darkyellow}{1 0.9 0}
Die Bindungsenergie im Tr�pfchenmodell setzt sich aus
folgenden Teilen zusammen:
\begin{itemize}
\item dem \rnode{b}{Oberfl�chenanteil}
\item Dem \rnode{a}{Volumenanteil},\\[1cm]
\def\xstrut{\vphantom{\frac{(A)^1}{(B)^1}}}
\begin{equation}
E =
\rnode[t]{ae}{\psframebox*[fillcolor=darkyellow,
  linestyle=none]{\xstrut a_vA}} +
\rnode[t]{be}{\psframebox*[fillcolor=lightgray,
  linestyle=none]{\xstrut -a_fA^{2/3}}} +
\rnode[t]{ce}{\psframebox*[fillcolor=green,
  linestyle=none]{\xstrut -a_c\frac{Z(Z-1)}{A^{1/3}}}} +
\rnode[t]{de}{\psframebox*[fillcolor=cyan,
  linestyle=none]{\xstrut -a_s\frac{(A-2Z)^2}{A}}} +
\rnode[t]{ee}{\psframebox*[fillcolor=yellow,
  linestyle=none]{\xstrut E_p}}
\end{equation}\\[0.25cm]
\item dem \rnode{c}{Coulomb-Anteil}
\item der \rnode{d}{Symmetrieenergie}
\item sowie einem \rnode{e}{Paarbildungsbeitrag}.
\end{itemize}

\nccurve[angleA=-90,angleB=90]{->}{a}{ae}
\nccurve[angleB=45]{->}{b}{be}
\nccurve[angleB=-90]{->}{c}{ce}
\nccurve[angleB=-90]{->}{d}{de}
\nccurve[angleB=-90]{->}{e}{ee}
\end{verbatim}

\end{document}